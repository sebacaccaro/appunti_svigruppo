\documentclass[a4paper,12pt]{article}

% Se vuoi che il pdf sia in formato mobile, decommenta la linea qui sotto e commenta la prima linea del codice
%\documentclass[1pt]{article}
\usepackage{enumitem}
\usepackage[paper size={90mm, 160mm},left=2mm,right=2mm,top=2mm,bottom=2mm,nohead]{geometry}
\usepackage{microtype}
\setlist[itemize]{leftmargin=*}


\usepackage{float}
\usepackage{url}
\usepackage{xcolor}
\usepackage{pdfpages}
\usepackage{graphicx}


%Comando per creare nuove definizioni stile blocco
\newcommand{\definition}[2]{
	\begin{table}[H]
	\centering
		\begin{tabular}{|p{0.9\linewidth}}
		\textbf{#1}\\ %Titolo della definzione
		#2\\%Testo della definizione
		\end{tabular}
	\end{table}
	\noindent
}

\newcommand{\df}[2]{\definition{#1}{#2}}

%%%%%%%%%%%%%%%%%%%%%%%%%%%%%%%%%%%%%%%%%%%%%%%%%%%%%%%%%%%%%%%%%%%%%
%      INSBOX --- macros for inserting pictures into paragraphs     %
%       Micha\l{} Gulczy\'nski, Szczecin, Jan 1996 / Feb 1998       %
%                     mgulcz@we.tuniv.szczecin.pl                   %
%%%%%%%%%%%%%%%%%%%%%%%%%%%%%%%%%%%%%%%%%%%%%%%%%%%%%%%%%%%%%%%%%%%%%
%
%  version 2.2
%
%  available macros:
%    * \InsertBoxC{anybox}
%        insert a centered box (use int _inside_ a paragraph)
%    * \InsertBoxL{after_line}{anybox}[correction]
%    * \InsertBoxR{after_line}{anybox}[correction]
%        insert a box in the left/right after specified number of lines;
%        correction specified in square brackets is optional;
%        both macros should be called _before_ a paragraph
%    * \MoveBelowBox
%        start a new paragraph just below the current frame
%
%  see the demo.tex file for more information
%

\catcode`\@ = 11
%
%  Margin between the text and the box:
\newdimen\@InsertBoxMargin
\@InsertBoxMargin = 2mm
%
%  definition of \ParShape, an inproved version of plain \parshape
%
\newcount\@numlines    % sum: m_1+...+m_n
\newcount\@linesleft   % counter used when reading lines of \ParShape
\def\ParShape{%
    \@numlines = 0
    \def\@parshapedata{ }% here we'll collect data for plain \parshape
    \afterassignment\@beginParShape
    \@linesleft
}%
\def\@beginParShape{%
    \ifnum \@linesleft = 0
      \let\@whatnext = \@endParShape
    \else
      \let\@whatnext = \@readnextline
    \fi
    \@whatnext
}%
\def\@endParShape{%
    \global\parshape = \@numlines \@parshapedata
}%
\def\@readnextline#1 #2 #3 {% #1 #2 #3 are: m_i, leftskip_i, rightskip_i
    \ifnum #1 > 0
      \bgroup  % I want to keep changes of \dimen0 and \count0 local
        \dimen0 = \hsize
        \advance \dimen0 by -#2  % \parshape requires left skip and
        \advance \dimen0 by -#3  % _length_of_line_ (not right skip!)
        \count0 = 0
        \loop
          \global\edef\@parshapedata{%
            \@parshapedata    % add to \@parshapedata:
            #2                % left skip
            \space            % a space
            \the\dimen0       % length of line
            \space            % another space
          }%
          \advance \count0 by 1
          \ifnum \count0 < #1
        \repeat
      \egroup
      \advance \@numlines by #1
    \fi
    \advance \@linesleft by -1
    \@beginParShape
}%
%
%  \InsertBoxC, \InsertBoxL, \InsertBoxR
%
\newbox\@boxcontent     % box containing the picture to be inserted
\newcount\@numnormal    % number of leading lines to typeset normally
\newdimen\@framewidth   % width of the frame
\newdimen\@wherebottom  % position of frame's bottom
\newif\if@byframe       % true if we are just beside the frame
\@byframefalse
%
%
\def\InsertBoxC#1{%
  \leavevmode
  \vadjust{
    \vskip \@InsertBoxMargin
    \hbox to \hsize{\hss#1\hss}
    \vskip \@InsertBoxMargin
  }%
}%
\def\InsertBoxL#1#2{%
  \@numnormal = #1
  \setbox\@boxcontent = \hbox{#2}%
  \let\@side = 0
  \futurelet \@optionalparameter \@InsertBox
}
\def\InsertBoxR#1#2{%
  \@numnormal = #1
  \setbox\@boxcontent = \hbox{#2}%
  \let\@side = 1
  \futurelet \@optionalparameter \@InsertBox
}%
\def\@InsertBox{%
  \ifx \@optionalparameter [
    \let\@whatnext = \@@InsertBoxCorrection
  \else
    \let\@whatnext = \@@InsertBoxNoCorrection
  \fi
  \@whatnext
}%
\def\@@InsertBoxCorrection[#1]{%
  \ifx \@side 0
    \@@InsertBox{#1}{0}{{\the\@framewidth} 0cm}%
  \else
    \@@InsertBox{#1}{1}{0cm {\the\@framewidth}}%
  \fi
}%
\def\@@InsertBoxNoCorrection{%
  \@@InsertBoxCorrection[0]%
}%
\def\@@InsertBox#1#2#3{%
  \MoveBelowBox
  \@byframetrue
  % \@wherebottom = \pagetotal + (\@numnormal * \baselineskip) +
  %                 (height of \@boxcontent) + (2 * \@InsertBoxMargin)
  \@wherebottom = \baselineskip
  \multiply \@wherebottom by \@numnormal
  \advance \@wherebottom by 2\@InsertBoxMargin
  \advance \@wherebottom by \ht\@boxcontent
  \advance \@wherebottom by \pagetotal
  % I have no idea why, but \InsertBox called at the top of a page
  % calculates space for the box one line too big
  \ifdim \pagetotal = 0cm
    \advance \@wherebottom by -\baselineskip  % ^ reduction
  \fi
  % add the correction
  \advance \@wherebottom by #1\baselineskip
  % \@framewidth = (width of \@boxcontent} + \@InsertboxMargin
  \@framewidth = \wd\@boxcontent
  \advance \@framewidth by \@InsertBoxMargin
  %
  \bgroup  % to keep changes of \dimen0 local
    % check if the box fits in the page
    \ifdim \pagetotal = 0cm
      \dimen0 = \vsize
    \else
      \dimen0 = \pagegoal
    \fi
    \ifdim \@wherebottom > \dimen0
      % print a warning message ...
      \immediate\write16{+--------------------------------------------------------------+}%
      \immediate\write16{| The box will not fit in the page. Please, re-edit your text. |}%
      \immediate\write16{+--------------------------------------------------------------+}%
      % ... and mark this place in document with a black box
      \vrule width \overfullrule
    \fi
  \egroup
  \prevgraf = 0
  % insert the box in the left (if #2 = 0) or in the right (if #2 = 1)
  \vbox to 0cm{%
    \dimen0 = \baselineskip
    \multiply \dimen0 by \@numnormal
    \advance \dimen0 by -\baselineskip
    \setbox0 = \hbox{y}%
    \vskip \dp0
    \vskip \dimen0
    \vskip \@InsertBoxMargin
    \ifnum #2 = 1
      \vtop{\noindent \hbox to \hsize{\hss \box\@boxcontent}}%
    \else
      \vtop{\noindent \box\@boxcontent}%
    \fi
    \vss
  }%
  % I have no idea why, but this is really necessary
  \vglue -\parskip
  \vskip -\baselineskip
  % each following paragraph needs to be formatted properly
  \everypar = {%
    % are we already below the bottom of the box?
    \ifdim \pagetotal < \@wherebottom
      % no...
      \bgroup  % to keep some changes local
        % let's calculate parameters for \ParShape
        \dimen0 = \@wherebottom
        \advance \dimen0 by -\pagetotal
        \divide \dimen0 by \baselineskip
        \count1 = \dimen0
        \advance \count1 by 1
        \advance \count1 by -\@numnormal
        \ifnum #2 = 1
          \ParShape = 3
                      {\the\@numnormal}   0cm   0cm
                      {\the\count1}       0cm   {\the\@framewidth}
                      1                   0cm   0cm
        \else
          \ParShape = 3
                      {\the\@numnormal}   0cm                  0cm
                      {\the\count1}       {\the\@framewidth}   0cm
                      1                   0cm                  0cm
        \fi
      \egroup
    \else
      % yes!
      \@restore@    % it's time to end everything
    \fi
  }%
  % this definition isn't very necessary --- just in case the paragraph
  % following \InsertBoxL or \InsertBoxR has fewer lines that the
  % first argument of the macro
  \def\par{%
      \endgraf
      \global\advance \@numnormal by -\prevgraf
      \ifnum \@numnormal < 0
        \global\@numnormal = 0
      \fi
      \prevgraf = 0
  }%
}%
%
%  call this macro to move the current position just below the
%  current frame
%
\def\MoveBelowBox{%
  \par
  \if@byframe
    \global\advance \@wherebottom by -\pagetotal
    \ifdim \@wherebottom > 0cm
      \vskip \@wherebottom
    \fi
    \@restore@
  \fi
}%
%
%  normal settings are as follows:
%
\def\@restore@{%
    \global\@wherebottom = 0cm
    \global\@byframefalse
    \global\everypar = {}%
    \global\let \par = \endgraf
    \global\parshape = 1 0cm \hsize
}%
%
%  someone told me that in LaTeX there is no \pageno counter;
%  the counterpart is \c@page
%
\ifx \documentclass \@Dont@Know@What@It@Is@
\else
  \let \pageno = \c@page
\fi


\catcode`\@ = 12

\newcommand{\lessonDate}[1]{\InsertBoxR{0}{\tiny{#1}}}

\newcommand{\E}{\`E\space}

\usepackage{listings}
\lstset{language=C++,
                keywordstyle=\color{blue},
                stringstyle=\color{red},
                commentstyle=\color{green},
                morecomment=[l][\color{magenta}]{\#}
}
\newcommand{\pyRef}[1]{{\color{blue}{DIP pg#1}}}

\begin{document}

\begin{titlepage}
\begin{center}
	\Large{\textbf{Appunti di Programmazione Avanzata}}
\vfill
\normalsize{Caccaro Sebastiano}\\
\normalsize{A.A.2019/2020}
\end{center}
\end{titlepage}

\tableofcontents


\clearpage


%Lunedi 7 ottobre
\lessonDate{7 Ottobre 2019}
\section{Informazioni sul corso}
\begin{itemize}
	\item\textbf{Login:}pa\\
	\item\textbf{Password:}PA+\#2009\#\\
	\item\textbf{Sito del corso:} \url{cazzola.di.unimi.it/pa.html}\\
\end{itemize}
Libri e cose del genere sono presenti nelle slide. L'esame è scritto e si svolge al computer.

\section{Python in breve}
\E un linguaggio di scripting multi-paradigma, quindi imperativo, object-oriented, e funzionale. \E interpretato, \textbf{object-based} (=ogni cosa è un oggetto) e tipizzato dinamicamente.\\
Nel corso si userà \textbf{Python 3}.

\subsection{Osservazioni su humanize.py}
\begin{itemize}
	\item Le variabili non hanno un tipo statico. Il tipo dinamico è attaccato all'oggetto che contiene la variabile.
	\item Supporta alcune strutture dati base di default. Ad esempio i dizionari (HashMap), liste ecc.
	\item Si può indentare con i tab o con gli spazi, molto meglio con gli spazi.
	\item Dopo la prima linea della funzione se commento con \texttt{``` commento ```}, posso fare un commento tipo javadoc.
	\item Come in javascript, tutto è un oggetto, anche le funzioni. Anche i primitivi sono degli oggetti. Come in JS, posso passare le funzioni come parametri.
	\item \texttt{\_\_name\_\_} assume il nome del file se viene importato come libreria, e assume il valore \texttt{\_\_main\_\_}
	\item Ci sono le \textbf{format string} che sono un po' come le template string di Javascript e PHP. Dalla 3.5 in poi non serve nemmeno il .format.
	\item Solo i tipi primitivi sono passati per valore, i tipi "complessi" sono passati per valore, quindi il contenuto è passato per riferimento.
	\item Qualsiasi riga di codice non protetta da if o dentro funzioni verrà eseguita. Non esiste un main.
\end{itemize}

\subsection{Altre Caratteristiche}
\begin{itemize}
	\item La keyword import mi permette di importare da altri moduli. Se voglio accedere a una funzione o campo dati specifico, faccio modulo.funzione. Posso accedere alla doc di un oggetto con \texttt{oggetto.\_\_doc\_\_}.
\end{itemize}

\subsection{Eccezioni}
Per lanciare un'eccezione si usa la keyworw \texttt{raise}. Si usano dei blocchi \texttt{try} - \texttt{except}.
%Metto qua lo snippet.
%\definition{Legge di Brooks}{Aggiungere personale ad un progetto in ritardo lo farà solo ritardare.}
\lessonDate{7 Ottobre 2019}

\subsection{Tipi di dato in Python}
I tipi di Python funzionano come in javascript, dove il tipo è associato all'oggetto.
Ci sono vari tipi numerici. Anche i primitivi sono oggetti, in quanto istanze di classi. 

\subsubsection{Numeri}
Si usa, ad esempio la classe \texttt{int}. I primi numeri di Python, circa i primi 256, sono implementati tramite \textbf{singleton}.\\
Alcuni comandi:
\begin{itemize}
	\item \texttt{type()} mi torna il tipo di una variabile
	\item \texttt{isinstance()} controlla il tipo di una variabile.
\end{itemize}
Posso rappresentare qualsiasi intero fino a infinito. I float sono precisi fino a 15 cifre decimali. Ci sono vari operatori, \pyRef{57}.

\subsubsection{Liste}
Sono trattate come degli array. Come javascript, posso avere più tipi di dato all'interno della stessa lista. Posso anche indicizzare al contrario. Esempio: \texttt{[-2]} accede al penultimo elemento della lista.\\
L'operatore di \textbf{slicing} è il seguente \texttt{[x:y]} dove \texttt{x} è mantenuto e \texttt{y} escluso. Posso sempre usare gli indici negativi. Se ometto uno dei due indici parto dall'inizio/fine. Creo sempre un nuovo oggetto.\\
Operatori: \pyRef{64}
\begin{itemize}
	\item \texttt{+}: Somma fra liste, creo un nuovo oggetto.
	\item \texttt{append}: appende un elemento alla fine della lista
	\item \texttt{extend:} aggiunge una lista ad un'altra lista
	\item \texttt{insert:} inserisce un elemento in una certa posizione nella lista
	\item \texttt{in}: del tipo \texttt{if x in list}, torna true e false
	\item \texttt{count} conta le istanze, in caso non ci sono, lancio eccezione
	\item \texttt{del}: \texttt{del list[2]} rimuovo per posizione
	\item \texttt{value}: \texttt{list.remove(3.13)} rimuove per valore
\end{itemize}

\subsubsection{Tuple}
Sono delle liste immutabili. Si scrivono con le tonde, non con le graffe.\\
Siccome non possono cambiare:
\begin{itemize}
 \item Sono più efficienti
 \item Posso usarle come chiave in un dizionario
\end{itemize}
Posso usare più o meno le stesse operazioni sulle liste, apparte quelle che apportano modifiche. Posso usare per assegnazioni multiple a variabili.

\subsubsection{Sets (insiemi)}
Struttura dati non ordinata con elementi univoci. Le creo come una lista ma con le graffe. Si crea un set vuoto con \texttt{set()}. Posso anche crearli da una lista.
Operatori:
\begin{itemize}
	\item \texttt{set.add} se possibile aggiunge un elemento al set
	\item \texttt{set1.update(set2)} unisce due set nel primo set
	\item \texttt{move discard}
	\item \texttt{union, difference ecc.} Tutte le operazioni matematiche sugli insiemi sono implementati. Potrebbe essere necessario ridefinire l'uguaglianza.
\end{itemize}

\subsubsection{Dizionari}
Insieme di coppie chiave-valore. La sintassi è praticamente quella del JSON. Sono praticamente delle HashMap. Il dizionario vuoto si scrive con \texttt{{}}. Sono disponibili i classici metodi che mi aspetterei.

\subsubsection{Stringe}
Si comportano come le liste. Uso len per la lunghezza. Posso delimitare le stringe con \texttt{"}, \texttt{'} e \texttt{```}.\\
Stringhe di formattazione (\pyRef{114}), meglio che guardo la doc. \E simile a js e php, ma tipo printf del C. Esiste anche il comando Template. Ci sono un'infinità di operazioni.


\subsection{Ricorsione}
La ricorsione in Python non è ricorsiva in coda. Quindi insomma lol, potrebbe dare qualche problema. L'iterazione è più efficiente. In Python ho massimo 1000 ricorsioni. Posso aggirare il problema, ma la ricorsione in Python \textbf{fa cagare}.

\lessonDate{14 Ottobre 2019}
\subsection{Comprehension}
Tramite una \textbf{comprehension} diamo una descrizione comprensiva di una lista di dati. \E come dare una formula matematica che descrive un insieme e una proprietà. \E una sorta di map, se vogliamo:
\begin{lstlisting}
>>> {elem:elem**2 for elem in range(1,10)}
{1: 1, 2: 4, 3: 9, 4: 16, 5: 25, 6: 36, 7: 49, 8: 64, 9: 81}
\end{lstlisting}
Posso usarlo per creare liste, dizionari, tuple ecc. \E da preferire rispetto ad usare un map, perchè è molto efficiente.\\
Posso utilizzare la comprehension anche per fare cose tipo un filter:
\begin{lstlisting}[caption=Espressione che crea tutti i quadrati perfetti da uno a cento]
>>> [elem for elem in range(1,100) if (int(elem**.5))**2 == elem]
[1, 4, 9, 16, 25, 36, 49, 64, 81]
\end{lstlisting}
Posso usarlo anche per mergiare set diversi, e fare varie figate. Ad esempio posso fare un prodotto cartesiano (doppio for) in modo figo.
\begin{lstlisting}
>>> {(x,y) for x in range(3) for y in range(5)}
{(0,1),(1,2),(0,0),(2,2),(1,1),(1,4),(0,2),(2,0),(1,3),
(2,3),(2,1),(0,4),(2,4),(0,3),(1,0)}
\end{lstlisting}
Lo svantaggio/problematica di questo metodo è che non mi da modi (facili) per interrompere l'esecuzione di questo finto for.



\end{document}