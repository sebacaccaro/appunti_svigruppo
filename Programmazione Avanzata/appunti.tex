\documentclass[a4paper,12pt]{article}

% Se vuoi che il pdf sia in formato mobile, decommenta la linea qui sotto e commenta la prima linea del codice
%\documentclass[1pt]{article}
\usepackage{enumitem}
\usepackage[paper size={90mm, 160mm},left=2mm,right=2mm,top=2mm,bottom=2mm,nohead]{geometry}
\usepackage{microtype}
\setlist[itemize]{leftmargin=*}


\usepackage{float}
\usepackage{url}
\usepackage{xcolor}
\usepackage{pdfpages}
\usepackage{graphicx}


%Comando per creare nuove definizioni stile blocco
\newcommand{\definition}[2]{
	\begin{table}[H]
	\centering
		\begin{tabular}{|p{0.9\linewidth}}
		\textbf{#1}\\ %Titolo della definzione
		#2\\%Testo della definizione
		\end{tabular}
	\end{table}
	\noindent
}

\newcommand{\df}[2]{\definition{#1}{#2}}

\input{insbox.tex}
\newcommand{\lessonDate}[1]{\InsertBoxR{0}{\tiny{#1}}}

\newcommand{\E}{\`E\space}

\usepackage{listings}
\lstset{language=C++,
                keywordstyle=\color{blue},
                stringstyle=\color{red},
                commentstyle=\color{green},
                morecomment=[l][\color{magenta}]{\#}
}
\newcommand{\pyRef}[1]{{\color{blue}{DIP pg#1}}}

\begin{document}

\begin{titlepage}
\begin{center}
	\Large{\textbf{Appunti di Programmazione Avanzata}}
\vfill
\normalsize{Caccaro Sebastiano}\\
\normalsize{A.A.2019/2020}
\end{center}
\end{titlepage}

\tableofcontents


\clearpage


%Lunedi 7 ottobre
\lessonDate{7 Ottobre 2019}
\section{Informazioni sul corso}
\begin{itemize}
	\item\textbf{Login:}pa\\
	\item\textbf{Password:}PA+\#2009\#\\
	\item\textbf{Sito del corso:} \url{cazzola.di.unimi.it/pa.html}\\
\end{itemize}
Libri e cose del genere sono presenti nelle slide. L'esame è scritto e si svolge al computer.

\section{Python in breve}
\E un linguaggio di scripting multi-paradigma, quindi imperativo, object-oriented, e funzionale. \E interpretato, \textbf{object-based} (=ogni cosa è un oggetto) e tipizzato dinamicamente.\\
Nel corso si userà \textbf{Python 3}.

\subsection{Osservazioni su humanize.py}
\begin{itemize}
	\item Le variabili non hanno un tipo statico. Il tipo dinamico è attaccato all'oggetto che contiene la variabile.
	\item Supporta alcune strutture dati base di default. Ad esempio i dizionari (HashMap), liste ecc.
	\item Si può indentare con i tab o con gli spazi, molto meglio con gli spazi.
	\item Dopo la prima linea della funzione se commento con \texttt{``` commento ```}, posso fare un commento tipo javadoc.
	\item Come in javascript, tutto è un oggetto, anche le funzioni. Anche i primitivi sono degli oggetti. Come in JS, posso passare le funzioni come parametri.
	\item \texttt{\_\_name\_\_} assume il nome del file se viene importato come libreria, e assume il valore \texttt{\_\_main\_\_}
	\item Ci sono le \textbf{format string} che sono un po' come le template string di Javascript e PHP. Dalla 3.5 in poi non serve nemmeno il .format.
	\item Solo i tipi primitivi sono passati per valore, i tipi "complessi" sono passati per valore, quindi il contenuto è passato per riferimento.
	\item Qualsiasi riga di codice non protetta da if o dentro funzioni verrà eseguita. Non esiste un main.
\end{itemize}

\subsection{Altre Caratteristiche}
\begin{itemize}
	\item La keyword import mi permette di importare da altri moduli. Se voglio accedere a una funzione o campo dati specifico, faccio modulo.funzione. Posso accedere alla doc di un oggetto con \texttt{oggetto.\_\_doc\_\_}.
\end{itemize}

\subsection{Eccezioni}
Per lanciare un'eccezione si usa la keyworw \texttt{raise}. Si usano dei blocchi \texttt{try} - \texttt{except}.
%Metto qua lo snippet.
%\definition{Legge di Brooks}{Aggiungere personale ad un progetto in ritardo lo farà solo ritardare.}
\lessonDate{7 Ottobre 2019}

\subsection{Tipi di dato in Python}
I tipi di Python funzionano come in javascript, dove il tipo è associato all'oggetto.
Ci sono vari tipi numerici. Anche i primitivi sono oggetti, in quanto istanze di classi. 

\subsubsection{Numeri}
Si usa, ad esempio la classe \texttt{int}. I primi numeri di Python, circa i primi 256, sono implementati tramite \textbf{singleton}.\\
Alcuni comandi:
\begin{itemize}
	\item \texttt{type()} mi torna il tipo di una variabile
	\item \texttt{isinstance()} controlla il tipo di una variabile.
\end{itemize}
Posso rappresentare qualsiasi intero fino a infinito. I float sono precisi fino a 15 cifre decimali. Ci sono vari operatori, \pyRef{57}.

\subsubsection{Liste}
Sono trattate come degli array. Come javascript, posso avere più tipi di dato all'interno della stessa lista. Posso anche indicizzare al contrario. Esempio: \texttt{[-2]} accede al penultimo elemento della lista.\\
L'operatore di \textbf{slicing} è il seguente \texttt{[x:y]} dove \texttt{x} è mantenuto e \texttt{y} escluso. Posso sempre usare gli indici negativi. Se ometto uno dei due indici parto dall'inizio/fine. Creo sempre un nuovo oggetto.\\
Operatori: \pyRef{64}
\begin{itemize}
	\item \texttt{+}: Somma fra liste, creo un nuovo oggetto.
	\item \texttt{append}: appende un elemento alla fine della lista
	\item \texttt{extend:} aggiunge una lista ad un'altra lista
	\item \texttt{insert:} inserisce un elemento in una certa posizione nella lista
	\item \texttt{in}: del tipo \texttt{if x in list}, torna true e false
	\item \texttt{count} conta le istanze, in caso non ci sono, lancio eccezione
	\item \texttt{del}: \texttt{del list[2]} rimuovo per posizione
	\item \texttt{value}: \texttt{list.remove(3.13)} rimuove per valore
\end{itemize}

\subsubsection{Tuple}
Sono delle liste immutabili. Si scrivono con le tonde, non con le graffe.\\
Siccome non possono cambiare:
\begin{itemize}
 \item Sono più efficienti
 \item Posso usarle come chiave in un dizionario
\end{itemize}
Posso usare più o meno le stesse operazioni sulle liste, apparte quelle che apportano modifiche. Posso usare per assegnazioni multiple a variabili.

\subsubsection{Sets (insiemi)}
Struttura dati non ordinata con elementi univoci. Le creo come una lista ma con le graffe. Si crea un set vuoto con \texttt{set()}. Posso anche crearli da una lista.
Operatori:
\begin{itemize}
	\item \texttt{set.add} se possibile aggiunge un elemento al set
	\item \texttt{set1.update(set2)} unisce due set nel primo set
	\item \texttt{move discard}
	\item \texttt{union, difference ecc.} Tutte le operazioni matematiche sugli insiemi sono implementati. Potrebbe essere necessario ridefinire l'uguaglianza.
\end{itemize}

\subsubsection{Dizionari}
Insieme di coppie chiave-valore. La sintassi è praticamente quella del JSON. Sono praticamente delle HashMap. Il dizionario vuoto si scrive con \texttt{{}}. Sono disponibili i classici metodi che mi aspetterei.

\subsubsection{Stringe}
Si comportano come le liste. Uso len per la lunghezza. Posso delimitare le stringe con \texttt{"}, \texttt{'} e \texttt{```}.\\
Stringhe di formattazione (\pyRef{114}), meglio che guardo la doc. \E simile a js e php, ma tipo printf del C. Esiste anche il comando Template. Ci sono un'infinità di operazioni.


\subsection{Ricorsione}
La ricorsione in Python non è ricorsiva in coda. Quindi insomma lol, potrebbe dare qualche problema. L'iterazione è più efficiente. In Python ho massimo 1000 ricorsioni. Posso aggirare il problema, ma la ricorsione in Python \textbf{fa cagare}.



\end{document}