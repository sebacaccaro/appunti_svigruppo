\documentclass[a4paper,12pt]{article}
\usepackage{float}
\usepackage{url}

\usepackage{float}
\usepackage{url}
\usepackage{xcolor}
\usepackage{pdfpages}
\usepackage{graphicx}


%Comando per creare nuove definizioni stile blocco
\newcommand{\definition}[2]{
	\begin{table}[H]
	\centering
		\begin{tabular}{|p{0.9\linewidth}}
		\textbf{#1}\\ %Titolo della definzione
		#2\\%Testo della definizione
		\end{tabular}
	\end{table}
	\noindent
}

\newcommand{\df}[2]{\definition{#1}{#2}}

\input{insbox.tex}
\newcommand{\lessonDate}[1]{\InsertBoxR{0}{\tiny{#1}}}

\newcommand{\E}{\`E\space}

\usepackage{listings}
\lstset{language=C++,
                keywordstyle=\color{blue},
                stringstyle=\color{red},
                commentstyle=\color{green},
                morecomment=[l][\color{magenta}]{\#}
}

\begin{document}

\begin{titlepage}
\begin{center}
	\Large{\textbf{Gestione dell'informazione Geospaziale}}
\vfill
\normalsize{Caccaro Sebastiano}\\
\normalsize{A.A.2019/2020}
\end{center}
\end{titlepage}
\tableofcontents

\clearpage


%Lunedi 7 ottobre
\lessonDate{7 Ottobre 2019}
\section{Introduzione}
\textbf{Geospaziale} tratta di dati sulla superficie terrestre. Posso trattare vari tipi di spazi, anche su unità di misura diverse, come lo spazio-tempo.

\subsection{Aspetti organizzativi}
\subsubsection{Obiettivi}
Concetti base:
\begin{itemize}
	\item Acquisizione dei dati
	\item Gestione dei dati: li mantengo in dei DBMS spaziali.
	\item Analisi dei dati, clustering (= raggruppare oggetti in base a criteri di omogeneità) ecc.
\end{itemize}

\subsubsection{Organizzazione del corso}
Il corso verrà così articolato:
\begin{itemize}
	\item Concetti base
	\item DBMS Spaziali
	\item Rappresentazione di oggetti in movimento
	\item Analisi dei dati spaziali.	
\end{itemize}

\textbf{Bisogna sapere PostgreSQL!}\\
L'esame consiste in un progetto più di un orale (potrebbe diventare una prova scritta all'ultima lezione del corso).\\
Il sito del corso è \url{homes.di.unimi.mdamiani/corsi/gig/}\\
\textbf{User}: gis7 \textbf{Pwd:} sql07sql

\subsubsection{Materiale}
\E presente un libro in formato PDF sul sito della docente.


\subsection{Concetti Base}
\subsubsection{Esempio di OpenStreetMaps}
OpenStreetMaps è una \textbf{mappa} aperta \textbf{collaborativa}, sostanzialmente un disegno. Ma non ci interessano i colori delle strade ecc. Cosa vuol dire Mappa Collaborativa?
\begin{itemize}
 \item \textbf{Mappa} è una banca dati che contiene informazioni geografiche coerenti, che poi vengono rappresentate tramite una mappa.
 \item \textbf{Collaborativa} la base di dati è modificabile da chiunque.\\
\end{itemize}


Costruire mappe è sempre stato altamente dispendioso, sopratutto per quanto riguarda l'acquisizione dei dati. OpenStreetMaps rende facile il reperimento e l'uso di dati spaziali.

\subsubsection{Spazio}
Parliamo di spazio geometrico con longitudine e latitudine, che vogliono una posizione e un sistema di riferimento. \E però molto più facile usare lo spazio cartesiano. Lo spazio simbolico invece rappresenta dei luoghi anche in base alla loro funzione (esempio cartine indoor). A seconda del tipo di spazio che sto analizzando, cambia anche la nozione di distanza. Ad esempio, come misuro la distanza in ambiente indoor?\\
Un oggetto può avere vari tipi di proprietà:
\begin{itemize}
\item Geometriche (forma)
\item Topografiche (i collegamenti, a mo di grafo)
\item Tematiche: caratteristiche, ad esempio i il numero di abitanti di un edificio
\end{itemize}

Ci possono essere movimenti di tipo:
\begin{itemize}
\item Continuo: ad esempio, movimento di palla nello spazio.
\item Discreto: ho un numero di posizioni finito, ad esempio posso essere in un dato momento solo sotto una cella telefonica.
\end{itemize}

%Possibile spostare a mappe
\E importantissimo poter visualizzare i dati. Altrimenti, non riesco a farmi un'idea di cosa ho in mano.

\subsubsection{Acquisizione dei dati}
Ci sono vari strumenti e metodi (arei, GPS, ecc.).\\
La posizione non è mai precisa al 100\%. I dati geospaziali dovrebbero essere sempre accompagnati anche dalla misura della loro incertezza.
\subsubsection{Analisi dei dati}
Posso analizzare i dati tramiti queri.
\subsubsection{Trattamento dei dati}
Tecnlogie:
\begin{itemize}
\item \textbf{Sistemi GIS} vedi slide: piattaforme software (= programmone) che mi permette di:
	\begin{itemize}
		\item Acquisire dati: digitalizzarli, controllarne la correttezza, integrare dati eterogenei (con fonti e caratteristiche diverse).
		\item Archiviare e accedere ai dati
		\item Trattare????? i dati
		\item Visualizzare dati
	\end{itemize}
\item \textbf{DBMS spaziali}: DBMS normali arricchiti con tipi e operazioni per dati spaziali. Praticamente sono SQL con estensioni spaziali.
	
\item Package specializzati.
\end{itemize}



%\definition{Legge di Brooks}{Aggiungere personale ad un progetto in ritardo lo farà solo ritardare.}

\end{document}